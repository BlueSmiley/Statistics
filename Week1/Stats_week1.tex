\documentclass[12pt]{article}
\usepackage[utf8]{inputenc}
\usepackage{amsmath}
\setlength{\parskip}{1em}

\title{Week 1 Questions}
\author{tprasad@tcd.ie 16326505}
\pagenumbering{gobble}
\begin{document}
\maketitle
\begin{flushleft}
Question 1.

(a) 10! since pick 10 choices with no replacement, each unique and no other limitations 
	\[ = 3628800 \]
	
(b) E and F must appear together so treat as one set taking two slots so now 9 slots. 
	9! but also order of E and F can swap so $9!*2!$ since 2 elements in set $\{E, F\}$ 
	\[ = 725760 \]

(c) Permutation formula. 6 slots so 6! but set of $\{N,N\}$ and $\{A,A,A\}$ has 2! and 3! ordering within the sets that don't matter. Hence $ \displaystyle \frac{6!}{(2!*3!)}$ 
 	\[ = 60 \]

(d) 3 slots. 5 unique choices. Assume order matters since letter arrangements. 
	Since 3 slots we don't care about the order of the 2 we don't pick. All elements are unique
	Hence $\displaystyle \frac{5!}{2!}$ 
	\[ = 60 \]
	
	Correction:
	
	Without any indication apparently order doesnt matter for this question so instead its
	$\displaystyle \binom{5}{3}$ if order doesn't matter 
	\[ = 10 \]
	
Question 2.

(a) $6^4$ Since 4 repetitions and 10 choices each time (with replacement) and order matters. 
	\[ = 1296 \]
	
(b) $\displaystyle \binom{4}{2}$ ways to pick exactly 2 slots being 3 and $5^{(4-2)}$ choices for each non-three slot
	hence $\displaystyle \binom{4}{2}*(5^2)$ \\
	\[ = 150 \]
	
(c)$ \displaystyle \binom{4}{2}*(5^2)$ where 2 3s + $\displaystyle \binom{4}{3}*(5)$ where 3 3s and $\displaystyle \binom{4}{4}$ where all 3s = $150 + 20 + 1$
	\[ = 171\]
	
Question 3.

(a) 8! ways to order cards but 4 sets with 2 duplicate elems each 
	$ \displaystyle \frac{8!}{(2!)^4} $ 
	\[ = 2520 \] 
	
(b) Both same set = $\displaystyle \binom{4}{1}$ = 4 ways.
	Different sets = $\displaystyle \binom{4}{2}$ = 6 ways
	Order doesnt matter
	Hence 6+4 = 10 unique combinations of two aces from the 8 cards
	\[ = 10 \]
	
	Correction:
	Both have to be different apparently so ignore the 4 extra choices and we get just 
	\[ = 6 \]
	
(c) $ \displaystyle \binom{2}{1} + \binom{2}{2} = 2 + 1 =3 $from same logic as before
	\[= 3 \]
	
	Alternative way to same answer: \\
	$\displaystyle \frac{4!}{2!*2!*2!}$ because $\displaystyle \frac{4!}{2!*2!}$ unique permutations of the 4 good cards. 
	But also we don't care about order of the final two cards we selected hence divide by 2! again.
	\[ = 3 \]
     
	

	
\end{flushleft}
\end{document}