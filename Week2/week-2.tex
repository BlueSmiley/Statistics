\documentclass[12pt]{article}
\usepackage[utf8]{inputenc}
\usepackage{amsmath}
\usepackage{listings}
\setlength{\parskip}{1em}

\title{Week 2 Questions}
\author{tprasad@tcd.ie 16326505}
\pagenumbering{gobble}
\begin{document}
\maketitle
\begin{flushleft}
Question 1.

(a) 6 choices each time. 
	If order of die rolls matters then $6^3= 216 $
	
(b) $5^3 = 125$ ways to not get a single 2. 
	$216 - 125 = 91$ = number of events where atleast single 2.
	$\frac{91}{216} = .421$ percentage of where atleast one 2.
		
(c)
\begin{minipage}{\textwidth}
\begin{lstlisting}[tabsize=4][language=Matlab]
function out = tripleDieRoll(reps)
	out = simulation(reps);
end

function out = dieRoll(sides,reps)
	out  = randi([1 sides],1,reps);
end


function prob = simulation(reps)
	count = 0;
	for index = 1:reps
		currentRolls = dieRoll(6,3);
		for throwno = 1:3
			if currentRolls(throwno) == 2
				count = count + 1;
				break;
			end
		end
	end
	prob = (count/reps)*100;
end
\end{lstlisting}
\end{minipage}
	
	call with tripleDieRoll(ARBIRTARY\_NUMBER\_OF\_SIMULATIONS)
	$42\% \pm 1\%$ with a number of 10 million simulations. 
		
(d) $6 + 6 + 5 = 17$ which is the only way to actually.
	get 17 but also can be in any order so
	\[\frac{3}{216} = 0.014\]
		
(e) Since purely a sum we don't really need to use conditional probability we can just consider 2 dice rolls that sum to $12-1 = 11$
	\[\{6,5\}, \{5,6\} \]
	\[\frac{2}{(6^2)} = 0.056\]
	
Question 2

(a) $ \frac{1}{6}$ chance of a 5 if 6 sided and $ \frac{1}{20}$ if 20 sided.
	$ \frac{1}{6}$ chance of a 1 and $\frac{5}{6}$ chance of anything else in first throw.
	\[\frac{1}{6}*\frac{1}{6} + \frac{5}{6}*\frac{1}{20} = .06894\]
	
(b) If 6 sided die then impossible hence
	\[\frac{1}{6}*0 + \frac{5}{6}*\frac{1}{20} =  0.0417\]


Question 3

	P(E|F)*P(F) = P(F|E)*P(E)
	
	Probability of brown hair P(F) = $.2*.4 + .6*1 = 0.68$ \newline
	Probability of being criminal given brown hair = ? \newline
	Probability of being criminal P(E) = .6 \newline
	Probability of brown hair given criminal = 1 \newline
	\[\frac{.6*1}{.68} = 0.882\]
	
Question 4

	P(Observe|Location) = given \newline
	P(Location) = given \newline
	P(Location| Observe) = unknown \newline
	P(Observation) = 100\% given infinite time \newline
	Assume that all prob stays constant till observation occurs and that all tiles have same P(Observation) 
	
	(P(O|L)*P(L))/P(O) = P(L|O)
	
	answer = 
	\begin{verbatim}
    	0.0375    0.0950    0.0375    0.0025
    	0.0025    0.0750    0.0475    0.0375
    	0.0005    0.0025    0.0750    0.0475
    	0.0005    0.0005    0.0050    0.0375
    	\end{verbatim}
    	
\begin{minipage}{\textwidth}
\begin{lstlisting}[tabsize=4][language=Matlab]
function resGrid = cell_tracker(locGrid, obsGivenLocGrid)
	[rowLen,colLen] = size(locGrid);
	if [rowLen,colLen] ~= size(obsGivenLocGrid)
		error("grid dimensions are different");
	end
	resGrid = zeros(rowLen,colLen);
	for i = 1:rowLen
		for j = 1:colLen
			% probability of observation is taken to be 100%
			% We assume no probability changes in time until observation
			resGrid(i,j) = calcCondLocProb(locGrid(i,j), ...
				obsGivenLocGrid(i,j),1);
		end
	end
end

function res = calcCondLocProb(locProb,obsGivenLocProb,obsProb)
	res = (obsGivenLocProb*locProb)/obsProb;
end
\end{lstlisting}
\end{minipage}

Iterates over both grids and applies formula given above to both grids
	
\end{flushleft}
\end{document}
	